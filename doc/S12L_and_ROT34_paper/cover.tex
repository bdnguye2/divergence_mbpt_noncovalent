\documentclass[12pt]{letter}

\usepackage[top=1.56in,left=1in,right=1in,bottom=1in,head=36pt,foot=50pt]{geometry}
% packages
\usepackage{graphicx}
\graphicspath{{$HOME/text/group/templates/letterhead/}} %$
\usepackage[usenames,dvipsnames,svgnames,table]{xcolor}
\usepackage{amsmath}
\usepackage{newtxtext,newtxmath}
\usepackage{fancyhdr}
\usepackage[colorlinks,citecolor=DeepPink4,linkcolor=DarkRed‌​,urlcolor=RoyalBlue]{hyperref}

% Manually set margins
\setlength{\parskip}{0.4em}
\setlength{\headsep}{1em}
\setlength{\textheight}{8.5in}


\usepackage{tabularx}


% set up fancyhdr
\fancypagestyle{uciletter}
{
  % headers
  \lhead{\hspace{-0.6in}\includegraphics[width=2.5in]{uci-stacked-wordmark-blue.pdf}}
  \rhead{Filipp Furche \\ {\small Professor}}

  % footers
  \lfoot{\hspace{-0.7in}\begin{minipage}[l]{1.in}\includegraphics[width=1.in]{seal-grey.pdf}\end{minipage}}
  \cfoot{}
  \rfoot{%\begin{minipage}[c]{5.in}
    \small
%      \hspace*{2in}
%    \vspace*{0.125in}
      \begin{tabular}{ll}
        Department of Chemistry \hspace*{0.5in} &(949) 824-5051 \\ 
        1102 Natural Sciences 2 & (949) 824-7672 \\ 
        Irvine, CA 92697-2025 &
        \sffamily \href{mailto:filipp.furche@uci.edu}{filipp.furche@uci.edu} \\ 
        & 
        \sffamily \href{http://ffgroup.chem.uci.edu/filipp}{ffgroup.chem.uci.edu/filipp}
        \\ 
      \end{tabular}}
%    \end{minipage}}

  \renewcommand{\headrulewidth}{0pt}
}
%\signature{\vspace{-3.5em} Philip J. Fry}
\signature{\vspace*{-3.5em}
  \begin{minipage}[l]{2in}
    \includegraphics[scale=0.25]{$HOME/Mail/uscan_blue} \\ %$
    Filipp Furche \\
    Professor of Chemistry
  \end{minipage} }
\pagestyle{uciletter}

\begin{document}

\date{\today}


\begin{letter}{Gustavo E. Scuseria \\
    Editor-in-Chief \\
    Journal of Chemical Theory and Computation \\
    Rice University \\
    }


\opening{Dear Professor Scuseria:}
\thispagestyle{uciletter}
I would like to submit the attached manuscript,

\begin{center}
  \begin{tabularx}{\textwidth}{lX}
    Title: & Divergence of Many-Body Perturbation Theory for Noncovalent
  Interactions of Large Molecules \\
    Authors: & Brian D. Nguyen, Guo P. Chen, Matthew M. Agee, Asbj{\"o}rn
    M. Burow, Matthew Tang, and Filipp Furche$^*$ \\
  \end{tabularx}
\end{center}
to your consideration for publication as a regular article in JCTC.

This work demonstrates that many-body perturbation theory (MBPT) diverges for
noncovalent interaction energies of most supramolecular complexes of
chemical interest. Our conclusions rely on basis-set extrapolated MBPT,
RPA, and CCSD(T) results for recently published benchmarks containing
complexes with 2-204 atoms with binding energies from less than 1 to 136
kcal/mol, combined with analytical results and explicit convergence radius
estimates. The latter are obtained from an asymptotic analysis in the
spirit of adiabatic connection density functional and symmetry-adapted
perturbation theory which cleanly separates dispersion and induction
effects and yields a non-perturbative definition of dispersion valid in
both the small molecule and the macroscopic limits. The physical origin
of the divergence is traced to incomplete ``electrodynamic'' screening
of the long-range Coulomb interaction between electrons in different
monomers by induced particle-hole pairs. An important practical
conclusion is that MP2 relative errors in binding energies increase
with system size at a rate of $\sim$ 1 \textperthousand\, per valence
electron.

This could be a useful paper for the JCTC community
because it debunks the conventional wisdom that MBPT works well for
``weak'' noncovalent interactions. Our results provide a compelling
explanation for the recently reported large errors of MBPT for binding
energies of large complexes. We also show that RPA overcomes these
limitations, yielding average relative errors of $\sim$ 5-10\% independent
of system size -- significantly less than previous studies using smaller
basis sets have reported. These results have important implications for
computational practice and for the development of \textit{ab initio}
and semi-empirical electronic structure methods.

The submitted version of the manuscript will be deposited on ChemRxiv.

We look forward to your and the Reviewers' comments.

\closing{Sincerely,}

\end{letter}
\end{document}
